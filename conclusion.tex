\section{Conclusion and Future Work}
\label{sec:future}

In this paper we presented Java Ranger as a path merging tool for Java. 
%
It works systematically by applying a series of transformations over a recovered statement from the CFG. This representation provides the benefit of modularity and visibility that promotes extension. 
%
Java Ranger has its own IR statement and its supports the construction of SSA for fields and arrays.
%
in addition Java Ranger supports inlining summaries of high order regions and it support the exploration of exceptional behavior via Single Path Cases. 
%
This maximizes portions of regions that could be summarized by Java Ranger. 
%
There are five main directions for future work we intend to do:
\begin{itemize}
\item Support Multiple Return Regions:  Here we want to support regions with multiple returns. 
%
We want to do this by supporting another transformation that would take a multiple return region and turns it into a single return region. 

\item Support of Test Case Generation: while statically summarizing regions gives dynamic symbolic execution a performance boost to explore more paths efficiently, generating test cases that covers all summarized branches is one of the fundamental roles of dynamic symbolic execution that is currently unsupported. 
%
This is one of the extension that we intend to investigate in our future work. 
%

\item Heuristics for Instantiation: as shown in the experiments section, instantiating regions might not always be the best option in terms of performance.
%
For instance, region instantiation could in fact turn concrete values into symbolic values thus increasing the number of explored paths. 
%
This is particularly problematic if that happened for a small region on an early branch point, and the symbolic variable is later used in multiple conditions.
%
This calls for a study of the best heuristics that would provide a decision procedure as to when instantiation a region.

\item Using Incremental Solver: this seems intuitive for two main reasons: firstly because as we go down one path the query sent to the solver is incrementally defined for every branch and yet it is repeatedly being sent to the solver. 
%
And secondly because using Java Ranger tend to have relatively complex formulas to solve due to summarization. 
%
Supporting incremental solver is one direction for our future work. 
%
\end{itemize}


