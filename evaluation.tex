\section{Evaluation}
\label{sec:results}
\subsection{Experimental Setup}
We implemented the above mentioned transformations as a wrapper around the Symbolic PathFinder~\cite{spf} tool.
%
To make use of region summaries in Symbolic PathFinder, we use an existing feature of SPF named \textit{listener}.
%
A listener is a method defined within SPF that is called for every bytecode instruction executed by SPF.
%
Java Ranger adds a path merging listener to SPF that, on every instruction, checks (1) if the instruction involves
checking a symbolic condition, and (2) if Java Ranger has a pre-computed static summary that begins at that
instruction\rq s bytecode offset.
%
If both of these conditions are satisfied, Java Ranger instantiates the multi-path region summary corresponding to that
bytecode offset by reading inputs from and writing outputs to the stack and the heap.
%
It then conjuncts the instantiated region summary with the path condition and resumes symbolic execution at the
bytecode offset of the end of the region.
%
Our implementation, named Java Ranger, wraps around SPF and can be configured to run in four modes.
%
(1) In mode 1, it runs vanilla SPF without any path merging enabled.
%
(2) In mode 2, it summarizes multi-path regions only and instantiates them if they are encountered.
%
(3) In mode 3, it summarizes and instantiates methods, by utilizing high ordder regions, along with multi-path regions as done in mode 2.
%
(4) In mode 4, it uses single-path cases along with multi-path region and method summaries used in mode 3.
%
\subsection{Evaluation}
\begin{table}[]
    \centering
    \begin{tabular}{|c|c|c|c|c|}
        \hline
        \begin{tabular}[c]{@{}c@{}}Benchmark\\ name\end{tabular} & \begin{tabular}[c]{@{}c@{}}Static analysis\\ time (msec)\end{tabular} & \begin{tabular}[c]{@{}c@{}}\# regions\\ summarized\end{tabular} & \begin{tabular}[c]{@{}c@{}}\# methods\\ summarized\end{tabular} & \begin{tabular}[c]{@{}c@{}}max. branch\\ depth\end{tabular} \\ \hline
        WBS                                                      & 13118                                                                 & 3507                                                            & 2203                                                            & 5                                                           \\ \hline
        TCAS                                                     & 13830                                                                 & 3505                                                            & 2209                                                            & 4                                                           \\ \hline
        Replace                                                  & 7020                                                                  & 119                                                             & 11                                                              & 10                                                          \\ \hline
    \end{tabular}
    \caption{Results of applying our static analysis on a few benchmarks}
    \label{table:static-analysis}
\end{table}
% Please add the following required packages to your document preamble:
% \usepackage{booktabs}
\begin{table}[]
    \begin{tabular}{@{}ccccccccc@{}}
        \toprule
        \begin{tabular}[c]{@{}c@{}}Bench\\ mark\\ Name\end{tabular} & \begin{tabular}[c]{@{}c@{}}Java\\ Ranger\\ mode\end{tabular} & \begin{tabular}[c]{@{}c@{}}\# \\ exec\\ paths\end{tabular} & \begin{tabular}[c]{@{}c@{}}run\\ time\\ (msec)\end{tabular} & \begin{tabular}[c]{@{}c@{}}\# \\ solver\\ queries\end{tabular} & \begin{tabular}[c]{@{}c@{}}solver\\ time\\ (msec)\end{tabular} & \begin{tabular}[c]{@{}c@{}}\# \\ inst.\\ regs\end{tabular} & \begin{tabular}[c]{@{}c@{}}\# \\ inst.\\ methods\end{tabular} & \# inst. \\ \midrule
        WBS-1step                                                   & 1                                                            & 24                                                         & 273                                                         & 46                                                             & 53                                                             & 0                                                          & 0                                                             & 0        \\ \midrule
        WBS-1step                                                   & 2                                                            & 1                                                          & 548                                                         & 0                                                              & 0                                                              & 6                                                          & 0                                                             & 6        \\ \midrule
        WBS-1step                                                   & 3                                                            & 1                                                          & 582                                                         & 0                                                              & 0                                                              & 6                                                          & 0                                                             & 6        \\ \midrule
        WBS-1step                                                   & 4                                                            & 1                                                          & 663                                                         & 6                                                              & 65                                                             & 6                                                          & 0                                                             & 6        \\ \midrule
        TCAS-SR-1step                                               & 1                                                            & 392                                                        & 4115                                                        & 2798                                                           & 3792                                                           & 0                                                          & 0                                                             & 0        \\ \midrule
        TCAS-SR-1step                                               & 2                                                            & 18                                                         & 999                                                         & 74                                                             & 428                                                            & 19                                                         & 0                                                             & 107      \\ \midrule
        TCAS-SR-1step                                               & 3                                                            & 1                                                          & 512                                                         & 0                                                              & 0                                                              & 4                                                          & 28                                                            & 4        \\ \midrule
        TCAS-SR-1step                                               & 4                                                            & 1                                                          & 619                                                         & 4                                                              & 72                                                             & 4                                                          & 28                                                            & 4        \\ \midrule
        Replace                                                     & 1                                                            & 1715                                                       & 3470                                                        & 5482                                                           & 2763                                                           & 0                                                          & 0                                                             & 0        \\ \midrule
        Replace                                                     & 2                                                            & 1080                                                       & 3.0E+04                                                     & 1.30E+04                                                       & 2.5E+04                                                        & 17                                                         & 0                                                             & 977      \\ \midrule
        Replace                                                     & 3                                                            & 632                                                        & 3.7E+04                                                     & 10259                                                          & 2.9E+04                                                        & 24                                                         & 113                                                           & 806      \\ \midrule
        Replace                                                     & 4                                                            & 345                                                        & 1.1E+05                                                     & 7020                                                           & 1.0E+05                                                        & 26                                                         & 122                                                           & 501      \\ \midrule \midrule
        TCAS-1step                                                  & 1                                                            & 392                                                        & 4698                                                        & 2798                                                           & 4287                                                           & 0                                                          & 0                                                             & 0        \\ \midrule
        TCAS-2steps                                                 & 1                                                            & 1.5E+05                                                    & 2.2E+06                                                     & 1.1E+06                                                        & 2.1E+06                                                        & 0                                                          & 0                                                             & 0        \\ \midrule
        TCAS-1step                                                  & 4                                                            & 178                                                        & 6848                                                        & 1719                                                           & 5349                                                           & 13                                                         & 0                                                             & 445      \\ \midrule
        TCAS-2steps                                                 & 4                                                            & 3.2E+04                                                    & 1.4E+06                                                     & 3.1E+05                                                        & 1.2E+06                                                        & 13                                                         & 0                                                             & 8.0E+04  \\ \midrule
        TCAS-SR-2steps                                              & 1                                                            & 1.5E+05                                                    & 1.7E+06                                                     & 1.1E+06                                                        & 1.7E+06                                                        & 0                                                          & 0                                                             & 0        \\ \midrule
        TCAS-SR-2steps                                              & 4                                                            & 1                                                          & 643                                                         & 8                                                              & 106                                                            & 4                                                          & 56                                                            & 8        \\ \midrule
        TCAS-SR-3steps                                              & 4                                                            & 1                                                          & 704                                                         & 12                                                             & 116                                                            & 4                                                          & 84                                                            & 12       \\ \midrule
        TCAS-SR-10steps                                             & 4                                                            & 1                                                          & 1087                                                        & 40                                                             & 354                                                            & 4                                                          & 280                                                           & 40       \\ \midrule
        WBS-2steps                                                  & 1                                                            & 576                                                        & 775                                                         & 1150                                                           & 418                                                            & 0                                                          & 0                                                             & 0        \\ \midrule
        WBS-3steps                                                  & 1                                                            & 1.4E+04                                                    & 9806                                                        & 2.8E+04                                                        & 8364                                                           & 0                                                          & 0                                                             & 0        \\ \midrule
        WBS-4steps                                                  & 1                                                            & 3.3E+05                                                    & 2.2E+05                                                     & 6.6E+05                                                        & 1.9E+05                                                        & 0                                                          & 0                                                             & 0        \\ \midrule
        WBS-5steps                                                  & 1                                                            & 8.0E+06                                                    & 5.3E+06                                                     & 1.2E+07                                                        & 4.4E+06                                                        & 0                                                          & 0                                                             & 0        \\ \midrule
        WBS-2steps                                                  & 4                                                            & 1                                                          & 971                                                         & 20                                                             & 401                                                            & 15                                                         & 0                                                             & 20       \\ \midrule
        WBS-3steps                                                  & 4                                                            & 1                                                          & 1344                                                        & 35                                                             & 769                                                            & 16                                                         & 0                                                             & 35       \\ \midrule
        WBS-4steps                                                  & 4                                                            & 1                                                          & 1919                                                        & 50                                                             & 1235                                                           & 16                                                         & 0                                                             & 50       \\ \midrule
        WBS-5steps                                                  & 4                                                            & 1                                                          & 2165                                                        & 65                                                             & 1523                                                           & 16                                                         & 0                                                             & 65       \\ \midrule
        WBS-6steps                                                  & 4                                                            & 1                                                          & 3239                                                        & 80                                                             & 2479                                                           & 16                                                         & 0                                                             & 80       \\ \midrule
        WBS-10steps                                                 & 4                                                            & 1                                                          & 3895                                                        & 140                                                            & 3163                                                           & 16                                                         & 0                                                             & 140      \\ \bottomrule
    \end{tabular}
    \caption{Java Ranger Performance on WBS, TCAS, TCAS-SR, and Replace}
    \label{table:results}
\end{table}
%
In order to evaluate the performance of Java Ranger, we used the following benchmarking programs commonly used to evaluate symbolic
execution performance.
%
(1) Wheel Brake System (WBS)~\cite{yang2014directed} is a synchronous reactive
component developed to make aircraft brake safely when taxing, landing, and during a rejected take-off.
%
(2) Traffic Collision Avoidance System (TCAS) is part of a suite of programs commonly referred to as the Siemens
suite~\cite{siemens-benchmarks}. TCAS is a system that maintains altitude separation between aircraft to avoid mid-air
collisions.
%
(3) Replace is another program that\rq s part of the Siemens suite. Replace searches for a pattern in a given input and
replaces it with another input string.
%
We used the translation of the Siemens suite to Java as made available by Wang et al.~\cite{dgse}.
%
We also manually created a variant of TCAS by converting regions of code with return statements on
every execution path to regions of code with a single return statement at the merge point of the region.
%
We refer to this variant of TCAS as TCAS-SR (TCAS with Single Returns).
%
We also created variants of WBS, TCAS, and TCAS-SR by running them for multiple steps.
%

We ran WBS, TCAS, TCAS-SR, and Replace using Java Ranger and first report the result of static analysis on these
three benchmarks in Table~\ref{table:static-analysis}.
%
The number of static regions summarized for WBS and TCAS is dominated by region summaries found in the Java standard
library used by WBS and TCAS.
%
Replace doesn't use the Java standard library, which is reflect in Table~\ref{table:static-analysis}.
%
We also report the most number of branches seen in our static regions with each benchmark.
%
Being able to instantiate static summaries of such regions potentially increases the path reduction we get from path-merging.

We present our results from instantiating region summaries in Table~\ref{table:results}.
%
Table~\ref{table:results} shows that Java Ranger achieves a significant improvement using path-merging with WBS.
%
The ``\# exec paths'' column is the number of execution paths required to completely explore the benchmark.
%
``runtime (msec)'' is the time taken for dynamic analysis (excludes static analysis time).
%
``\# inst. regs'' is the total number of regions that were instantiated at least once when running each benchmark.
%
``\# inst. methods'' is the number of higher-order regions that were used and ``inst.'' is the total number of
instantiations that was done when running the benchmark.
%
Table~\ref{table:results} shows that we achieve a significant improvement using path-merging with WBS.
%
Java Ranger outperforms SPF when running TCAS too but it doesn't allow path merging to summarize the entire program
to a single execution path, as in the case with WBS.
%
We manually analyzed TCAS and found the only barrier to our path-merging was the presence of several code
regions containing a return instruction on every execution path.
%
We manually performed a semantics-preserving transformation to the source code of TCAS to create another version, TCAS-SR,
that has a single return value at the end of all such multi-path regions.
%
We plan to automate this transformation in the future, see future work section~\ref{sec:futureWork}.
%
Table~\ref{table:results} shows the benefit of such a \textit{multiple-returns-to-single-return} transformation.
%
It allows Java Ranger to summarize the entire TCAS-SR program into a single execution path.
%
This is made possible by the use of higher-order regions as shown in Table~\ref{table:results}.
%
In mode 2, Java Ranger explores 18 execution paths with TCAS-SR, whereas in mode 3~(mode 2 + higher-order region support), it needs to only explore a single
execution path because it can instantiate 28 higher-order method summaries.
%
The bottom half of Table~\ref{table:results} also shows the benefits of path-merging with Java Ranger increase as we run
more steps of WBS, TCAS, and TCAS-SR.
%
We ran as many steps of each of these 3 benchmarks as possible with a 6 hour timeout using Java Ranger in modes
1 (runs Java Ranger without path merging) and 3 (runs Java Ranger with multi-path region summaries and higher-order region summaries).
%
While, in mode 1, Java Ranger can only run 5 steps of WBS in less than 6 hours, in mode 4 (path merging with multi-path regions, higher-order regions, and single-path cases),
Java Ranger can easily execute up to 10 steps in less than 4 seconds.
%
A similar benefit is seen with TCAS-SR where Java Ranger in mode 1 can only run 2 steps of TCAS-SR in 6 hours, but in
mode 4, it can easily run up to 10 steps of TCAS-SR in about 1 second.

However, the performance of Java Ranger when running Replace is quite different from that seen when running WBS, TCAS, and TCAS-SR.
%
While Java Ranger reduces the total number of execution paths with every increase in mode, the path reduction isn't as significant.
%
More importantly, it also makes an order of magnitude more solver calls thereby increasing the total runtime.

We further investigated this drop in performance of Java Ranger when running Replace by exploring the space of potential
regions to instantiate.
%
We obtained the list of regions that are instantiated at least once when running Replace with Java Ranger in mode 3~(using
high order and multi-path region summaries).
%
This step produced a list of 24 regions.
%
Next, we sampled the space of all possible subsets of this set of 24 regions by
%
(1) randomly enumerating the space up to 4000 subsets of all possible subsets of 24 regions,
%
(2) enumerating the space in the order of subset sizes up to all subsets of 3 regions from a set of 24 regions.
%
With both methods of enumeration, we searched for the least amount of time Java Ranger needs to instantiate at least one
region with Replace.
%
We found this least time to be 26.3 seconds which is still more than the 3.47 seconds Java Ranger needs to run
Replace in mode 1 (running vanilla SPF with no path merging).\vaibhav{update these numbers before submission}
%
In the fastest runs with at least one region instantiation in both methods of enumeration, Java Ranger finished
exploration with the same number of execution paths as that explored during mode 1.
%
The result of this analysis, combined with a significant increase in the number of solver queries with Java Ranger's
path-merging modes, points toward the conclusion that every instantiation of a multi-path region in Replace causes another branch to be
symbolic.
%
This conclusion is also in alignment with the observation made by Kuznetsov et al.~\cite{kuznetsov} that path merging can
sometimes have a net negative effect on the performance of symbolic execution.
%
We plan to integrate heuristics for estimating the side-effects of introducing new symbolic state as a result of
path merging on symbolic execution performance with Java Ranger in the future.
