\documentclass{acm_sen_article}

\usepackage{booktabs} % For formal tables
\usepackage{listings}
\usepackage{subfig}
\usepackage{url}
\usepackage{color}
\usepackage{authblk}
\renewcommand{\baselinestretch}{0.98}

% Copyright
%\setcopyright{none}
% \setcopyright{acmcopyright}
%\setcopyright{acmlicensed}
% \setcopyright{rightsretained}
%\setcopyright{usgov}
%\setcopyright{usgovmixed}
%\setcopyright{cagov}
%\setcopyright{cagovmixed}


% DOI
% \acmDOI{10.475/123_4}

% ISBN
% \acmISBN{123-4567-24-567/08/06}

%\acmPrice{15.00}


\begin{document}

\lstset{language=Java}

\lstdefinestyle{nonumbers}
{numbers=none}

\newcommand{\mike}[1]{\textcolor{red}{#1}}

\definecolor{mygreen}{rgb}{0,0.4,0}
\definecolor{mygray}{rgb}{0.5,0.5,0.5}
\definecolor{mymauve}{rgb}{0.58,0,0.82}
\lstset{ %
  backgroundcolor=\color{white},   % choose the background color; you
%must add \usepackage{color} or \usepackage{xcolor}
  basicstyle=\ttfamily\small,        % the size of the fonts that are used
%for the code
  basewidth = {.5em, 0.5em},
  breakatwhitespace=false,         % sets if automatic breaks should
%only happen at whitespace
  breaklines=true,                 % sets automatic line breaking
  captionpos=b,                    % sets the caption-position to bottom
  commentstyle=\color{mygreen},    % comment style
  deletekeywords={...},            % if you want to delete keywords from
%the given language
  escapeinside={\%*}{*)},          % if you want to add LaTeX within
%your code
  extendedchars=true,              % lets you use non-ASCII characters;
%for 8-bits encodings only, does not work with UTF-8
  frame=single,	                   % adds a frame around the code
  keepspaces=true,                 % keeps spaces in text, useful for
%keeping indentation of code (possibly needs columns=flexible)
  keywordstyle=\color{blue},       % keyword style
  language=C,                 % the language of the code
  otherkeywords={*,...},           % if you want to add more keywords to
%the set
  numbers=left,                    % where to put the line-numbers;
%possible values are (none, left, right)
  numbersep=5pt,                   % how far the line-numbers are from
%the code
  numberstyle=\tiny\color{black}, % the style that is used for the
%line-numbers
  rulecolor=\color{black},         % if not set, the frame-color may be
%changed on line-breaks within not-black text (e.g. comments (green
%here))
  showspaces=false,                % show spaces everywhere adding
%particular underscores; it overrides 'showstringspaces'
%  showstringspaces=false,          % underline spaces within strings
%only
  showtabs=false,                  % show tabs within strings adding
%particular underscores
  stepnumber=1,                    % the step between two line-numbers.
%If it's 1, each line will be numbered
  stringstyle=\color{mymauve},     % string literal style
  tabsize=2,	                   % sets default tabsize to 2 spaces
%  title=\lstname                   % show the filename of files included
%with \lstinputlisting; also try caption instead of title
  literate={->}{$\rightarrow$}{2}
           {α}{$\alpha$}{1}
           {δ}{$\delta$}{1}
}

\title{Veritesting in Symbolic Execution of Java}
\author[1]{}


\maketitle
\begin{abstract}
Path explosion problem is still the main obsticale against scaling up symbolic execution to industrial sized projects. 
%
One interesting resolution to the problem is \emph{Veritesting}, which represents regions of code as disjunctive formals over paths.
%
Unlike the C compiler that inlines functions in programs, Integrating veritesting with Java bytecode presents unique challenges: notably, incorporating non-local control jumps caused by runtime polymorphism, exceptions, native calls, and dynamic 3 class loading.
%
In this paper we present our robust implementation of Java based veritesting tool that supports dynamic dispatch. 
 
\end{abstract}

\keywords{multi-path symbolic execution; veritesting; Symbolic
PathFinder; static analysis}


%Path explosion problem is still the main obsticale against scaling up symbolic execution to industrial sized projects.
%%
%One interesting resolution to the problem is \emph{Veritesting}, which represents regions of code as disjunctive formals over paths.
%%
%Unlike the C compiler that inlines functions in programs, Integrating veritesting with Java bytecode presents unique challenges: notably, incorporating non-local control jumps caused by runtime polymorphism, exceptions, native calls, and dynamic 3 class loading.
%%
%In this paper we present our robust implementation of Java based veritesting tool that supports dynamic dispatch and
\section{Introduction}
\vaibhav{needs rewriting, assigned to anyone who can find the time to write it}

Symbolic execution is a popular analysis technique that performs non-standard execution of a program: data operations generate formulas over inputs, and the branch constrains along an execution path are combined into a predicate.
%
Originally developed in the 1970s~\cite{King1976,Clarke1976}, symbolic execution is a convenient building block for program analysis, since arbitrary query predicates can be combined with the logical program representation, and solutions to these constraints are program inputs illustrating the queried behavior.
%
Some of the many application of symbolic execution include
test generation~\cite{dart,cute}, equivalence checking~\cite{ramos,adaptorsynth}, vulnerability finding~\cite{driller,angr}, and protocol correctness checking~\cite{transport}.
%
Symbolic execution tools are available for many languages, including
CREST~\cite{BurnimS2008} for C source code, KLEE~\cite{CadarDE2008}
for C/C++ via LLVM, JDart~\cite{jdart2016} and Symbolic
PathFinder~\cite{spf} for Java, and S2E~\cite{ChipounovKC2012},
FuzzBALL~\cite{BabicMMS2011}, and angr~\cite{angr} for binary code.
%
 \mike{More here...explain the `ecosystem' - tools for different languages: KLEE, FuzzBall, Java Symbolic Pathfinder, ...}

Although symbolic analysis is a very popular technique, scalability is a substantial challenge for symbolic execution.
%
Dynamic state merging~\cite{kuznetsov} provides one way to
alleviate scalability challenges by opportunistically merging dynamic
symbolic executors, which can be performed on paths~\mike{Add std. cite} or on environments~\mike{FM paper from 2014 on Javascript?}.  
Other techniques include CEGAR/subsumption~\mike{Add references from ASE 2017 paper: More Effective Interpolations in Software Model Checking}.
 
%
Veritesting~\cite{veritesting} is a different recently proposed technique that can dramatically improve the performance of symbolic execution.  Rather than explicitly merge paths or check subsumption relationships, Veritesting simply encodes a local region of a program containing branches as a disjunctive region for symbolic analysis.  If any path within the region meets an exit point, then the disjunctive formula is satisfiable.  This often allows many paths to be collapsed into a single path involving the region.  
%
In previous work~\cite{veritesting}, bounded static code regions have been shown to find more bugs, and achieve more node and path coverage, when implemented at the X86 binary level for compiled C programs.
%
This provides motivation for investigating integration of introducing static regions with symbolic execution at the Java bytecode level.

\lstinputlisting[caption={An example to loop through a symbolic array with three execution paths through the loop body},
label={lst:v_ex}]{code_samples/VeritestingPerf.java}



%Symbolic Pathfinder~(SPF)~\cite{spf} is a tool that performs symbolic execution of Java bytecode.
%
%SPF is tightly integrated with Java PathFinder~(JPF)~\cite{jpf} and uses JPF extensions to replace concrete execution with symbolic execution.
%
We present an example demonstrating the potential benefit of integrating static code regions with SPF in Listing~\ref{lst:v_ex}.
%
The example checks if positive or negative integers occur more frequently in the
list~\textit{x}, and it contains a bug if \textit{x} contains an
equal number of positive and negative integers.
%
The three-way branch on lines 5, 6 causes the total number of execution
paths required to cover the \textit{for} loop to be $3^{\textit{len}}$.
%
However, this three-way branch can be combined into a multi-path region
and represented as a disjunctive predicate.
We present such predicates in SMT2 notation in
Listing~\ref{lst:v_ex_smt2} assuming \textit{x} to contain two symbolic
integers named \textit{x0} and \textit{x1}~(\textit{len} equals 2).
%
The updates to \textit{sum} in the two loop iterations are captured by
\textit{sum0} and \textit{sum1}.
%
Using such predicates to represent the three-way branch on lines 5, 6 of
Listing~\ref{lst:v_ex} allows us to have only one execution path through
the loop body.
%
Figure~\ref{fig:v_ex_plot} shows a comparison of the number of execution
paths explored to find the bug on line 11 of Listing~\ref{lst:v_ex}.
%
The exponential speed-up from our predicates, representing a multi-path
region, allows us to find
the bug using just three test cases.

Unfortunately, as originally proposed, Veritesting would be unable to create a static region for this loop because it involves non-local control jumps (the calls to the \texttt{get} methods).  This is not an impediment for compiled C code, as the C compiler will usually automatically inline the code for short methods such as \texttt{get}.  However, Java has an {\em open world} assumption, and most methods are {\em dynamically dispatched}, meaning that the code to be run is not certain until resolved at runtime, so the compiler is unable to perform these optimizations.

In Java, programs often consist of many small methods that are dynamically dispatched, leading to poor performance for n\"aive implementations of bounded static regions.  Thus, to be successful, we must be able to inject the static regions associated with the calls into the dispatching region.  We call such regions {\em higher order} as they require a region as an argument and can return a region that may need to be further interpreted.
Given support for such regions, we can make analysis of programs such as~\ref{lst:v_ex} trivial for large loop depths.  In our experiments, we demonstrate 100x speedups on several models (in general, the more paths contained within a program, the larger the speedup) over the unmodified Java SPF tool using this approach.
 
%
\lstinputlisting[caption={SMT2 representation of multi-path execution in
Listing ~\ref{lst:v_ex} using \textit{len} = 2}, label={lst:v_ex_smt2}, language=lisp]{code_samples/ex.smt2.snippet}
%
\begin{figure}[]
\caption{Comparing number of execution paths from Listing~\ref{lst:v_ex} using vanilla SPF and SPF with static unrolling}
\label{fig:v_ex_plot}
\includegraphics[width=\columnwidth]{figures/veritesting_example_semilogy}
\end{figure}
%
\subsection{Motivating Example}
\vaibhav{assigned to Vaibhav}

\section{Preliminaries}
\subsection{CFG}
A control flow graph (CFG) in computer science is a representation, using graph notation, of all paths that might be traversed through a program during its execution.
In a control flow graph each node in the graph represents a basic block, i.e. a straight-line piece of code without any jumps or jump targets; jump targets start a block, and jumps end a block. Directed edges are used to represent jumps in the control flow. There are, in most presentations, two specially designated blocks: the entry block, through which control enters into the flow graph, and the exit block, through which all control flow leaves.

\subsection{SPF}
Symbolic PathFinder (SPF) \cite{spf} combines symbolic execution with model checking and constraint solving for test case generation. In this tool, programs are executed on symbolic inputs representing multiple concrete inputs. Values of variables are represented as numeric constraints, generated from analysis of the code structure. These constraints are then solved to generate test inputs guaranteed to reach that part of code. Essentially SPF performs symbolic execution for Java programs at the bytecode level. Symbolic PathFinder uses the analysis engine of the Ames JPF model checking tool (i.e. jpf-core) \cite{jpf}.


\subsection{Veritesting}
Veritesting\cite{veritesting} is a technique that reduces the number of paths that need to be explored by avoiding unnecessary forking. 
%
Veritesting identifies regions of if-statements that can be statically explored and captured in a logical formula (usually as disjunction of formulas representing different branches in if-statement). 
%
We will call this formula VeriFormula which captures possible different execution paths. 
%
The result of this process is a VeriFormula, which can be submitted to a SMT solver. 
%
If the formula is satisfiable, then there is some path through the code region that reaches the exit point. In this case, dynamic symbolic execution is resumed after updating the PC with the VeriFormula and also updating symbolic values of variable if needed. 

\section{Architecture}

\subsection{Shared Expressions}
%Sharing implementation needs to be fixed. Show this using the TestSharing example
%
Veritesting causes regions of code to be executed using static symbolic execution.
%
Symbolic formulas representing the static symbolic execution are then gathered at the exit points of the region and added to the path expression and symbolic store of dynamic symbolic execution.
%
This causes large disjunctive formulas to be substituted and reused
multiple times, necessitating the use of techniques like hash
consing~\cite{hashconsing}, or its variants such as maximally-shared
graphs~\cite{babic}.%, or using expression caching~\cite{green}.

\subsection{Complex Expressions}
%Engineering issue \#2: Need to have complex expressions, talk about how Comparators cannot be used anywhere below the top-level operator
During exploration, SPF creates conjunctions of expressions and adds them to its {\em PathCondition} to determine satisfiability of paths.
%
These expressions are allowed to have a \textit{Comparator}~(a comparison operator such as !=) as the top-level operator; however, comparison and Boolean operators are not allowed in sub-expressions.  Thus, the current set of SPF expressions is insufficiently expressive to represent the disjunctive formulas required for multi-path regions. To support that use used Green\cite{green} AST.

\section{Experiment}
\vaibhav{assigned to Vaibhav}

We would like to measure the performance of \tool\ against the baseline of single-path exploration using Java Symbolic
PathFinder.  This can be examined in several dimensions: the wall-clock time of the solving process, the number of paths
explored.  In addition, we would also like to gather metrics about the regions themselves, in order to better understand
where static regions are effective.

Therefore, we investigate the following research questions:

\begin{description}
\item[RQ1:] How much do higher-order static regions (HOSR) improve the performance of symbolic execution?
\item[RQ2:] How much do HOSRs reduce the number of paths explored?
\item[RQ3:] How do HOSRs affect the number and expense of calls to the SMT solver?
\item[RQ4:] How does the size of computed HOSRs affect the performance of the approach?
\end{description}

For each question, we examine three different configurations of \tool: a version that creates simple regions (one
branch) only (\tool-SR), one that creates complex regions with multiple branches but no non-local jumps (\tool-CR), and
one that operates over higher-order complex regions containing non-local jumps (\tool-CR+HO). This allows us to examine
(at a coarse level) how each feature impacts the experimental results.

\subsection{Experimental Setup}

\mike{Information here about benchmark models and machine configuration}

\mike{The benchmarks should be a superset of at least one previous paper, and better yet, multiple papers}. 
\section{Future Work}
\label{sec:futureWork}
While \tool\ was able to significantly outperform SPF in a majority of our benchmarks, there are a few directions
along which it can further be extended.
%
Java Ranger attempts to perform path merging as aggressively as possible.
%
This path merging strategy doesn't optimize towards making fewer solver calls.
%
We plan to work towards implementing heuristics that can measure the effect of path merging on the rest of the program.
%
\tool\ currently lacks support for symbolic object and array references.
%
Supporting these would require integrating our implementation with
SPF\rq s lazy initialization~\cite{spf} to allow symbolic object references to be part of a region.

While path-merging gives dynamic symbolic execution a performance boost to explore more paths
efficiently, generating test cases that cover all branches is one of the most useful applications of dynamic
symbolic execution.
%
This is an example of an application of symbolic execution that, if applied as-is, would have to undo the benefits of
path-merging.
%
We intend to extend \tool\ towards test generation for merged paths in the future by targeting test generation
towards a coverage criterion such as Modified Condition/Decision Coverage.

While path merging can potentially allow symbolic execution to explore interesting parts of a program sooner, the
effect of path merging on search strategies, such as depth-first search and breadth-first search commonly used with
symbolic execution, remains to be investigated.
%
We plan to explore the integration of such guidance heuristics with path merging in the future.
%
Java Ranger can summarize methods and regions in Java standard libraries.
%
This creates potential for automatically constructing summaries of standard libraries so that symbolic execution
can prevent path explosion originating from standard libraries.
\section{Related Work}
%Talk about veritesting in Mayhem, Angr.
%
The original idea for veritesting was presented by Avgerinos et al.~\cite{veritesting}.
%
They implemented veritesting on top of MAYHEM~\cite{mayhem}, a system for finding bugs at the X86 binary level which uses symbolic execution.
%
Their implementation demonstrated dramatic performance improvements and allowed them to find more bugs, and have better node and path coverage.
%
Veritesting has also been integrated with another binary level symbolic execution engine named {\tt angr}~\cite{angr}.
%
Veritesting was added to {\tt angr} with similar goals of statically and selectively merging paths to mitigate path explosion.
%
However, path merging from veritesting integration with {\tt angr} caused complex expressions to be introduced which overloaded the constraint solver.
%
Using the Green~\cite{green} solver may alleviate such problems when implementing veritesting with SPF.
%
Another technique named \textit{MultiSE} for merging symbolic execution states incrementally
was proposed by Sen et al.~\cite{multise}.
%
MultiSE computes a set of guarded symbolic expressions for every
assignment and does not require identification of points where
previously forked dynamic symbolic executors need to be merged.
%
MultiSE complements predicate construction for multi-path regions beyond
standard exit points~(such as
\textit{invokevirtual},~\textit{invokeinterface},~\textit{return}
statements).
%
Combining both techniques, while a substantial implementation effort, has the potential
to amplify the benefits from both techniques.
%This observation became even more apparent for longer paths.
%
%Integrating veritesting with SPF may expose a similar trade-off.
%
%But we expect this problem to be less severe for when doing static symbolic execution of Java bytecode because Java bytecode instructions have fewer behaviors to be statically analyzed than X86 architecture instructions.

% %Talk about other symbolic execution performance improvements.
% %mentioned in Christopher Kruegel's ISSTA keynote talk as Static Analysis support
% Other static analysis techniques also provide support for dynamic symbolic execution.
% %
% Loop-extended symbolic execution introduced partial loop summarization by having symbolic variables that represent the number of times each loop executes.
% %
% This technique allowed symbolic variables to incorporate loop dependent effects along with data dependencies from program inputs.
%
%Value Set Analysis~\cite{vsa} is another static analysis technique that can potentially benefit dynamic symbolic execution.
%%
%Value set analysis uses an abstract domain to find an over-approximated set of values that registers or abstract locations may have at program points.
%%
%Value set analysis can help dynamic symbolic execution resolve ranges without solving constraints.
%VSA can resolve ranges without solving constraints, thereby, finding applications in computing all possible write targets during a memory write operation.
%Code summarization (Dodo)
%  - automatically (and statically) summarize effect of loops / functions
%VSA - value set analysis
%  - resolve ranges (and conditionals) without solving constraints

%Talk about TamiFlex, and how using the same technique as TamiFlex is one way to solve veritesting challenges in Java bytecode.
%
%Other techniques for static analysis at the Java bytecode level can also benefit dynamic symbolic execution.
%
Finding which reflective method call is being used, or handling dynamic class loading are known problems for static analysis tools.
%
TamiFlex~\cite{tamiflex} provides an answer that is sound with respect to a set of previously seen program runs.
%
Integrating veritesting runs into similar problems, and using techniques from TamiFlex would allow static predicate construction beyond exit points caused by reflection or dynamic class loading. 

\section{Conclusion and Future Work}
\label{sec:future}

\soha{we need something for the conclusion}
There are five main directions for future work we intend to do:
\begin{itemize}
\item Support Multiple Return Regions:  Here we want to support regions with multiple returns. 
%
We want to do this by supporting another transformation that would take a multiple return region and turns it into a single return region. 

\item Support of Test Case Generation: while statically summarizing regions gives dynamic symbolic execution a performance boost to explore more paths efficiently, generating test cases that covers all summarized branches is one of the fundamental roles of dynamic symbolic execution that is currently unsupported. 
%
This is one of the extension that we intend to investigate in our future work. 
%

\item Heuristics for Instantiation: as shown in the experiments section, instantiating regions might not always be the best option in terms of performance.
%
For instance, region instantiation could in fact turn concrete values into symbolic values thus increasing the number of explored paths. 
%
This is particularly problematic if that happened for a small region on an early branch point, and the symbolic variable is later used in multiple conditions.
%
This calls for a study of the best heuristics that would provide a decision procedure as to when instantiation a region.

\item Using Incremental Solver: this seems intuitive for two main reasons: firstly because as we go down one path the query sent to the solver is incrementally defined for every branch and yet it is repeatedly being sent to the solver. 
%
And secondly because using Java Ranger tend to have relatively complex formulas to solve due to summarization. 
%
Supporting incremental solver is one direction for our future work. 
\end{itemize}



Other improvements:
\begin{enumerate}
\item Simplification of the Symbolic PathFinder constraint mechanism

\item Interpolation-based path subsumption checks  (c.f.: "More Effective Interpolations in Software Model Checking" - ASE 2017")

\end{enumerate}




\bibliographystyle{ACM-Reference-Format}
\bibliography{references}

\end{document}
