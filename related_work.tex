\section{Related Work}
\vaibhav{in progress}

Path explosion is a major cause of scalability limitations for
symbolic execution, so an appealing direction for optimization is to
combine the representations of similar execution paths, which we refer
to generically as {\em path merging.}
%
If a symbolic execution tool already concurrently maintains objects
representing multiple execution states, a natural approach is to merge
these states, especially ones with the same control-flow location.
%
Hansen et al.~\cite{HansenSS2009} explored this technique but found
mixed results on its benefits.
%
Kuznetsov et al.~\cite{kuznetsov} developed new algorithms and
heuristics to control when to perform such state merging profitably.
%
A larger departure in the architecture of symbolic execution systems
is the MultiSE approach proposed by Sen et al.~\cite{multise}, which
represents values including the program counter with a two-level
guarded structure, in which the guard expressions are optimized with
BDDs.
%
The MultiSE approach achieves effects similar to state merging
automatically, and provides some architectural advantages such as in
representing values that are not supported by the SMT solver.

Another approach to achieve path merging is to statically summarize
regions that contain branching control flow.
%
This approach was proposed by Avgerinos et al.~\cite{veritesting} and
dubbed ``veritesting'' because it pushes symbolic execution further
along a continuum towards all-paths techniques used in verification.
%
A veritesting-style technique is a convenient way to add path merging
to a symbolic execution system that maintains only one execution
state, which is one reason we chose it for use in SPF.
%
Avgerinos et al. designed and implemented their veritesting system
MergePoint for the application of binary-level symbolic execution for
bug finding.
%
They found that veritesting provided a dramatic performance
improvement, allowing their system to find more bugs and have better
node and path coverage in a given exploration time.
%
The static regions used by MergePoint are intra-procedural, but they
can have any number of ``transition points'' at which control can be
returned to regular symbolic execution.
%
Avgerinos et al. do not provide details about how MergePoint
represents memory accesses or integrates them with veritesting, though
since MergePoint was built as an extension of the same authors' Mayhem
system, it may reuse techniques such as symbolic representation of
loads from bounded memory regions~\cite{mayhem}.

The veritesting approach has been integrated with another binary level
symbolic execution engine named {\tt angr}~\cite{angr}.
%
However angr's authors found that their veritesting implementation did
not provide an overall improvement over their dynamic symbolic
execution baseline: though veritesting allowed some new crashes to be
found, they observed that giving more complex symbolic expressions
slowed down the SMT solver enough that total performance was degraded.
%
We have also observed complex expressions to be a potential cost of
veritesting, but we believe that optimizations of the SMT solver
interface and potentially heuristics to choose when to use veritesting
can allow it to be a net asset.

% %Talk about other symbolic execution performance improvements.
% %mentioned in Christopher Kruegel's ISSTA keynote talk as Static Analysis support
% Other static analysis techniques also provide support for dynamic symbolic execution.
% %
% Loop-extended symbolic execution introduced partial loop summarization by having symbolic variables that represent the number of times each loop executes.
% %
% This technique allowed symbolic variables to incorporate loop dependent effects along with data dependencies from program inputs.
%
%Value Set Analysis~\cite{vsa} is another static analysis technique that can potentially benefit dynamic symbolic execution.
%%
%Value set analysis uses an abstract domain to find an over-approximated set of values that registers or abstract locations may have at program points.
%%
%Value set analysis can help dynamic symbolic execution resolve ranges without solving constraints.
%VSA can resolve ranges without solving constraints, thereby, finding applications in computing all possible write targets during a memory write operation.
%Code summarization (Dodo)
%  - automatically (and statically) summarize effect of loops / functions
%VSA - value set analysis
%  - resolve ranges (and conditionals) without solving constraints

%Talk about TamiFlex, and how using the same technique as TamiFlex is one way to solve veritesting challenges in Java bytecode.
%
%Other techniques for static analysis at the Java bytecode level can also benefit dynamic symbolic execution.
%
Finding which reflective method call is being used, or handling dynamic class loading are known problems for static analysis tools.
%
TamiFlex~\cite{tamiflex} provides an answer that is sound with respect to a set of previously seen program runs.
%
Integrating veritesting runs into similar problems, and using techniques from TamiFlex would allow static predicate construction beyond exit points caused by reflection or dynamic class loading. 
